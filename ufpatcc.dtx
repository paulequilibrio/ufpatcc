% \iffalse meta-comment
%
% Copyright (C) 2017 by Paulo Alexandre Aquino da Costa <contato@pauloalexandre.com>
%
% This work may be distributed and/or modified under the
% conditions of the LaTeX Project Public License (LPPL), either
% version 1.3c of this license or (at your option) any later
% version.  The latest version of this license is in the file:
%
% http://www.latex-project.org/lppl.txt
%
% This work is "maintained" (as per LPPL maintenance status) by
% Paulo Alexandre Aquino da Costa.
%
% This work consists of the file ufpatcc.dtx and a Makefile.
% Running "make" generates the derived files README, ufpatcc.pdf and ufpatcc.cls.
% Running "make inst" installs the files in the user's TeX tree.
% Running "make install" installs the files in the local TeX tree.
%
% \fi
%
% \iffalse
%<*internal>
\iffalse
%</internal>
%<*readme>
# ufpatcc
Classe LaTeX para trabalhos de conclusão de curso na Universidade Federal do Pará (UFPA).
|
-------:| -----------------------------------------------------------------
ufpatcc:| Classe LaTeX para trabalhos de conclusão de curso na Universidade Federal do Pará (UFPA)
 Author:| Paulo Alexandre Aquino da Costa
 E-mail:| contato@pauloalexandre.com
License:| Released under the LaTeX Project Public License v1.3c or later
    See:| http://www.latex-project.org/lppl.txt

Classe criada para facilitar a escrita de trabalhos de conclusão de curso
na Universidade Federal do Pará (UFPA).
Esta classe extende a classe \LaTeX{} |report|,
acrescentado algumas opções, comandos e ambientes para a formatação automática das
partes pré-textuais do trabalho de conclusão como, por exemplo, a capa, a folha de rosto,
a ficha catalográfica e a folha de aprovação.

%</readme>
%<*internal>
\fi
\def\nameofplainTeX{plain}
\ifx\fmtname\nameofplainTeX\else
  \expandafter\begingroup
\fi
%</internal>
%<*install>
\input docstrip.tex
\keepsilent
\askforoverwritefalse
\preamble

Copyright (C) 2018 by Paulo Alexandre Aquino da Costa <contato@pauloalexandre.com>

This work may be distributed and/or modified under the
conditions of the LaTeX Project Public License (LPPL), either
version 1.3c of this license or (at your option) any later
version.  The latest version of this license is in the file:

http://www.latex-project.org/lppl.txt

This work is "maintained" (as per LPPL maintenance status) by
Paulo Alexandre Aquino da Costa.

\endpreamble
\postamble

-------:| -----------------------------------------------------------------
ufpatcc:| Classe LaTeX para trabalhos de conclusão de curso na Universidade Federal do Pará
 Author:| Paulo Alexandre Aquino da Costa
 E-mail:| contato@pauloalexandre.com
License:| Released under the LaTeX Project Public License v1.3c or later
    See:| http://www.latex-project.org/lppl.txt

This work consists of the file ufpatcc.dtx and a Makefile.
Running "make" generates the derived files README, ufpatcc.pdf and ufpatcc.cls.
Running "make inst" installs the files in the user's TeX tree.
Running "make install" installs the files in the local TeX tree.

\endpostamble

\usedir{tex/latex/ufpatcc}
\generate{
  \file{\jobname.cls}{\from{\jobname.dtx}{class}}
}
%</install>
%<install>\endbatchfile
%<*internal>
\usedir{source/latex/ufpatcc}
\generate{
  \file{\jobname.ins}{\from{\jobname.dtx}{install}}
}
\nopreamble\nopostamble
\usedir{doc/latex/ufpatcc}
\generate{
  \file{README.txt}{\from{\jobname.dtx}{readme}}
}
\ifx\fmtname\nameofplainTeX
  \expandafter\endbatchfile
\else
  \expandafter\endgroup
\fi
%</internal>
% \fi
%
% \iffalse
%<*driver>
\ProvidesFile{ufpatcc.dtx}
%</driver>
%<class>\NeedsTeXFormat{LaTeX2e}[2017/01/01]
%<class>\ProvidesClass{ufpatcc}
%<*class>
    [2018/01/31 v0.3 Classe LaTeX para
    trabalhos de conclusao de curso na
    Universidade Federal do Para (UFPA)]
%</class>
%<*driver>
\documentclass{ltxdoc}
\usepackage[a4paper,margin=25mm,left=50mm,nohead]{geometry}
\usepackage[numbered]{hypdoc}
\usepackage[utf8]{inputenc}
\usepackage[T1]{fontenc}
\usepackage[brazil]{babel}
\usepackage{indentfirst}
\usepackage{fancyvrb}
\setlength{\parindent}{0.75cm}
\setlength{\parskip}{0.5ex plus 0.5ex minus 0.2ex}

% \edef\glossary@prologue\section*{{Change Histor}}\markboth{{Change Histo}}{{Change Histoy}}

% \EnableCrossrefs
\CodelineIndex
\RecordChanges
\begin{document}
  \DocInput{\jobname.dtx}
\end{document}
%</driver>
% \fi
%
% \GetFileInfo{\jobname.dtx}
% \DoNotIndex{\newcommand,\newenvironment}
%
%\changes{v0.1}{2017/02/18}{Primeira versão publicada}
%\changes{v0.2}{2018/01/31}{Pequena correção}
%\changes{v0.3}{2018/01/31}{Melhorias no exemplo de uso}
%
%
%\title{Classe \textsf{ufpatcc}\thanks{Este documento descreve a versão
%   \fileversion, publicada em \filedate.}}
%\author{Paulo Alexandre Aquino da Costa\\ \texttt{contato@pauloalexandre.com}}
%\date{\filedate{} \fileversion}
%
%\maketitle
%
%
% \begin{abstract}
% Classe criada para facilitar a escrita de trabalhos de conclusão de curso
% na Universidade Federal do Pará (UFPA).
% Esta classe é uma extensão a classe \LaTeX{} |report|,
% acrescentado algumas opções, comandos e ambientes para a formatação automática das
% partes pré-textuais do trabalho de conclusão como a capa e a folha de rosto, por exemplo.
% \end{abstract}
%
% \section{Como usar esta classe}
%
% \subsection{O arquivo de configuração}
%   Esta classe necessita de um arquivo de configuração com extensão |.cfg|
%   e o mesmo nome que o arquivo |.tex| que está usando esta classe. Por exemplo,
%   se o arquivo principal do trabalho for |seunome.tex| e tiver o seguinte conteúdo:
%   \begin{Verbatim}[frame=single,fontsize=\small,gobble=6]
%     \documentclass{ufpatcc}
%     \begin{document}
%       \capa
%       \folhaderosto
%       \chapter{Introdução}
%       \bibliographystyle{IEEEtran.bst}
%       \referencias[Referências]{referencias}
%     \end{document}
%   \end{Verbatim}
%   \noindent{}será necessário ter, no mesmo diretório, um arquivo |seunome.cfg|,
%   o qual pode até mesmo estar vazio, mas deve existir para que a compilação ocorra corretamente.
%
%   Um exemplo deste arquivo |.cfg| podem ser visto a seguir:
%   \begin{Verbatim}[frame=single,fontsize=\small,gobble=6]
%     \autor{Fulano da Silva}{SILVA, Fulano da}
%     \grau{Engenheiro}
%     \logo{figures/logo}
%     \instituto{Instituto de Tecnologia}
%     \faculdade{Faculdade de Engenharia da Computação e Telecomunicações}
%     \curso{Engenharia da Computação}
%     \titulo{Agilidade no desenvolvimento de software}
%     \palavraschave{desenvolvimento de software, agilidade}
%     \keywords{software development, agile}
%     \orientador{Profª. Drª. Beltrana Ferreira}
%     \coorientador{Prof. Dr. Sicrano dos Santos}
%     \diretordafaculdade{Prof. Dr. Fulano Beltrano de Matos}
%     \membrodabanca{Profª. Drª. Beltrana Ferreira, Universidade Federal do Pará}
%     \membrodabanca{Prof. Dr. Sicrano dos Santos, Universidade Federal do Pará}
%     \membrodabanca{Prof. Dr. Albert Einstein, Institute for Advanced Study}
%     \ano{2018}
%   \end{Verbatim}
%
% \clearpage
% Os comandos apresentados a seguir podem ser utilizados no arquivo de configuração.
% Cada um deles será apresentado com uma breve descrição seguida de um exemplo de uso.
%
% \DescribeMacro{\autor}
%   |\autor|\marg{nome}
%   O comando |\autor| possui dois parâmetros obrigatórios.
%   O primeiro parâmetro deve ser o nome completo do autor.
%   O segundo parâmetro deve ser o nome do autor em formato para citação,
%   conforme exemplo abaixo.
%   \par{} Deve ser especificado, pois não possui valor default.
%   \begin{Verbatim}[frame=single,gobble=6]
%     \autor{Fulano da Silva}{SILVA, Fulano da}
%   \end{Verbatim}
%
% \DescribeMacro{\tipodetrabalho}
%   O comando |\tipodetrabalho| possui apenas um parâmetro obrigatório.
%   Exemplos de valores:
%   \begin{itemize}
%     \item Trabalho de Conclusão de Curso (default)
%     \item Dissertação de Mestrado
%     \item Tese de Doutorado
%   \end{itemize}
%   \begin{Verbatim}[frame=single,gobble=6]
%     \tipodetrabalho{Tese de Doutorado}
%   \end{Verbatim}
%
% \DescribeMacro{\nivel}
%   O comando |\nivel| possui apenas um parâmetro obrigatório.
%   Exemplos de valores:
%   \begin{itemize}
%     \item Graduação (default)
%     \item Mestrado
%     \item Doutorado
%   \end{itemize}
%   \begin{Verbatim}[frame=single,gobble=6]
%     \nivel{Doutorado}
%   \end{Verbatim}
%
% \DescribeMacro{\grau}
%   O comando |\grau| possui apenas um parâmetro obrigatório,
%   que identifica o grau a ser obtido quando da conclusão do curso.
%   \par{} Deve ser especificado, pois não possui valor default.
%   \begin{Verbatim}[frame=single,gobble=6]
%     \grau{Engenheiro}
%   \end{Verbatim}
%
% \DescribeMacro{\logo}
%   O comando |\logo| possui apenas um parâmetro obrigatório,
%   que é o caminho (relativo ou absoluto) do arquivo com a logomarca da universidade.
%   O arquivo pode ter extensão |.png|, |.jpg| ou |.pdf|,
%   entretanto não é necessário especificar a extensão do arquivo neste comando.
%   \par{} Este é um comando opcional. Não possui valor default.
%   \begin{Verbatim}[frame=single,gobble=6]
%     \logo{figures/logo}
%   \end{Verbatim}
%
% \DescribeMacro{\universidade}
%   O comando |\universidade| possui apenas um parâmetro obrigatório,
%   que define o nome da universidade.
%   \par{} O valor default é |Universidade Federal do Pará|.
%   \begin{Verbatim}[frame=single,gobble=6]
%     \universidade{Universidade do Estado do Pará}
%   \end{Verbatim}
%
% \DescribeMacro{\instituto}
%   O comando |\instituto| possui apenas um parâmetro obrigatório,
%   que define o nome do instituto do curso.
%   \par{} O valor default é |Instituto de Tecnologia|.
%   \begin{Verbatim}[frame=single,gobble=6]
%     \instituto{Instituto de Ciências Exatas e Naturais}
%   \end{Verbatim}
%
% \DescribeMacro{\faculdade}
%   O comando |\faculdade| possui apenas um parâmetro obrigatório,
%   que define o nome da faculdade do curso.
%   \par{} Deve ser especificado, pois não possui valor default.
%   \begin{Verbatim}[frame=single,gobble=6]
%     \faculdade{Faculdade de Engenharia Elétrica}
%   \end{Verbatim}
%
% \DescribeMacro{\curso}
%   O comando |\curso| possui apenas um parâmetro obrigatório,
%   que define o nome do curso.
%   \par{} Deve ser especificado, pois não possui valor default.
%   \begin{Verbatim}[frame=single,gobble=6]
%     \curso{Engenharia da Computação}
%   \end{Verbatim}
%
% \DescribeMacro{\titulo}
%   O comando |\titulo| possui apenas um parâmetro obrigatório,
%   que define o título do trabalho.
%   \par{} Deve ser especificado, pois não possui valor default.
%   \begin{Verbatim}[frame=single,gobble=6]
%     \titulo{Agilidade no desenvolvimento de software}
%   \end{Verbatim}
%
% \DescribeMacro{\subtitulo}
%   O comando |\subtitulo| possui apenas um parâmetro obrigatório,
%   que define o título do trabalho.
%   \par{} Este é um comando opcional. Não possui valor default.
%   \begin{Verbatim}[frame=single,gobble=6]
%     \subtitulo{estudo comparativo}
%   \end{Verbatim}
%
% \DescribeMacro{\palavraschave}
%   O comando |\palavraschave| possui apenas um parâmetro obrigatório,
%   que define as palavras-chave do trabalho.
%   \par{} Deve ser especificado, pois não possui valor default.
%   \begin{Verbatim}[frame=single,gobble=6]
%     \palavraschave{desenvolvimento de software, agilidade}
%   \end{Verbatim}
%
% \DescribeMacro{\keywords}
%   O comando |\keywords| possui apenas um parâmetro obrigatório,
%   que define as keywords (palavras-chave em inglês) do trabalho.
%   \par{} Deve ser especificado, pois não possui valor default.
%   \begin{Verbatim}[frame=single,gobble=6]
%     \keywords{software development, agile}
%   \end{Verbatim}
%
% \DescribeMacro{\orientador}
%   O comando |\orientador| possui apenas um parâmetro obrigatório,
%   que define o nome completo do orientador do trabalho, com sua titulação.
%   \par{} Deve ser especificado, pois não possui valor default.
%   \begin{Verbatim}[frame=single,gobble=6]
%     \orientador{Profª. Drª. Beltrana Ferreira}
%   \end{Verbatim}
%
% \DescribeMacro{\coorientador}
%   O comando |\coorientador| possui apenas um parâmetro obrigatório,
%   que define o nome completo do coorientador do trabalho, com sua titulação.
%   \par{} Este é um comando opcional. Não possui valor default.
%   \begin{Verbatim}[frame=single,gobble=6]
%     \coorientador{Prof. Dr. Sicrano dos Santos}
%   \end{Verbatim}
%
% \DescribeMacro{\diretordafaculdade}
%   O comando |\diretordafaculdade| possui apenas um parâmetro obrigatório,
%   que define o nome completo do diretor da faculdade, com sua titulação.
%   \par{} Este é um comando opcional. Não possui valor default.
%   \begin{Verbatim}[frame=single,gobble=6]
%     \diretordafaculdade{Prof. Dr. Fulano Beltrano de Matos}
%   \end{Verbatim}
%
% \DescribeMacro{\membrodabanca}
%   O comando |\membrodabanca| possui apenas um parâmetro obrigatório,
%   que define o nome completo de um membro da banca de avaliação do trabalho, com sua titulação.
%   Este comando pode ser utilizado mais de uma vez, pois seu valor não será sobrescrito,
%   mas sim adicionado a uma lista.
%   O parâmetro deve seguir o formato mostrado abaixo: nome com titulação
%   seguido por uma vírgula e depois o nome da instituição à qual o membro
%   é vinculado.
%   \par{} Deve ser especificado, pois não possui valor default.
%   \begin{Verbatim}[frame=single,fontsize=\small,gobble=6]
%     \membrodabanca{Profª. Drª. Beltrana Ferreira, Universidade Federal do Pará}
%     \membrodabanca{Prof. Dr. Sicrano dos Santos, Universidade Federal do Pará}
%     \membrodabanca{Prof. Dr. Albert Einstein, Institute for Advanced Study}
%   \end{Verbatim}
%
% \DescribeMacro{\cidade}
%   O comando |\cidade| possui apenas um parâmetro obrigatório.
%   \par{} O valor default é |Belém|.
%   \begin{Verbatim}[frame=single,gobble=6]
%     \cidade{Recife}
%   \end{Verbatim}
%
% \DescribeMacro{\estado}
%   O comando |\estado| possui apenas um parâmetro obrigatório.
%   \par{} O valor default é |Pará|.
%   \begin{Verbatim}[frame=single,gobble=6]
%     \estado{Pernambuco}
%   \end{Verbatim}
%
% \DescribeMacro{\ano}
%   O comando |\ano| possui apenas um parâmetro obrigatório.
%   \par{} O valor default é |o ano atual|.
%   \begin{Verbatim}[frame=single,gobble=6]
%     \ano{2016}
%   \end{Verbatim}
%
%
% \subsection{Comandos}
%
%   Todos os comando descritos nessa seção utilizam os valores fornecidos
%   no arquivo de configuração para gerarem as estruturas do trabalho.
%
% \DescribeMacro{\capa}
%   Insere a capa.
%
% \DescribeMacro{\folhaderosto}
%   Insere a folha de rosto.
%
% \DescribeMacro{\fichacatalografica}
%   Insere a ficha catalográfica.
%   A classificação do trabalho e o CDD devem ser obtidos
%   com o bibliotecário responsável pela biblioteca.\par
%   \hspace*{-\parindent}|\fichacatalografica|\marg{biblioteca}\marg{classificacao}\marg{CDD}
%   \begin{Verbatim}[frame=single,fontsize=\small,gobble=6]
%     \fichacatalografica{Biblioteca Prof. Dr. Clodoaldo Beckmann}
%                        {1. Desenvolvimento de software 2. Métodos ágeis I. Título}
%                        {21.ed. 005.1}
%   \end{Verbatim}
%
% \DescribeMacro{\folhadeaprovacao}
%   Insere a folha de aprovação.\par
%   \hspace*{-\parindent}|\folhadeaprovacao|\marg{data da defesa}
%   \begin{Verbatim}[frame=single,gobble=6]
%     \folhadeaprovacao{28/02/2018}
%   \end{Verbatim}
%
% \DescribeMacro{\referencias}
%   Insere a lista de referências, que podem vir de vários arquivos |.bib|.\par
%   \hspace*{-\parindent}|\referencias|\oarg{título da lista}\marg{arquivo(s) .bib sem extensão}
%   \begin{Verbatim}[frame=single,gobble=6]
%     \referencias[Referências Bibliográficas]{referencias1,referencias2}
%   \end{Verbatim}
%
% \subsection{Ambientes}
%
% A classe \texttt{\jobname{}} disponibiliza os ambientes a seguir.
% Embora os exemplos apresentados aqui sejam curtos, é possível
% inserir múltiplos parágrafos em todos os ambientes.
% Os textos dentro dos ambientes serão inseridos na posição em que
% os ambientes forem usados no trabalho.
%
% \clearpage
%
% \DescribeEnv{dedicatoria}
%   Insere a dedicatória.
%   \begin{Verbatim}[frame=single,gobble=6]
%     \begin{dedicatoria}
%       Dedico esse trabalho aos meus pais.
%     \end{dedicatoria}
%   \end{Verbatim}
%
% \DescribeEnv{agradecimentos}
%   Insere os agradecimentos.
%   \begin{Verbatim}[frame=single,gobble=6]
%     \begin{agradecimentos}
%       Agradeço aos meus colegas de turma e ao meu orientador.
%     \end{agradecimentos}
%   \end{Verbatim}
%
% \DescribeEnv{epigrafe}
%   Insere a epígrafre.
%   \begin{Verbatim}[frame=single,gobble=6]
%     \begin{epigrafe}
%       ``A maior de todas as vitórias é sobre si mesmo.''\par
%       Siddhartha Gautama
%     \end{epigrafe}
%   \end{Verbatim}
%
% \DescribeEnv{resumo}
%   Insere o resumo.
%   \begin{Verbatim}[frame=single,gobble=6]
%     \begin{resumo}
%       Desenvolvimento ágil de software.
%     \end{resumo}
%   \end{Verbatim}
%
% \DescribeEnv{abstract}
%   Insere o abstract.
%   \begin{Verbatim}[frame=single,gobble=6]
%     \begin{abstract}
%       Agile software development.
%     \end{abstract}
%   \end{Verbatim}
%
%
% \subsection{Opções da classe}
%   Esta classe implementa apenas duas opções:
%
%   \begin{description}
%     \item[times] altera a fonte do trabalho para uma similar a Times New Roman.
%     \item[arial] altera a fonte do trabalho para uma similar a Arial.
%   \end{description}
%
%   Além destas duas opções, também estão disponíveis todas as demais opções
%   da classe |report|. Observe que essa classe já carrega a classe |report| com
%   as opções |a4paper| e |12pt| ativadas.
%
%\StopEventually{^^A
%  \clearpage
%  \PrintChanges
% \iffalse
%  \PrintIndex
% \fi
%}
%
% \section{Implementação da classe}
% \setlength{\parindent}{0cm}
%    \begin{macrocode}
%<*class>
%    \end{macrocode}
% ifs para controle das opções da classe
%    \begin{macrocode}
\newif\if@times
\newif\if@arial
%    \end{macrocode}
% Opções da classe
%    \begin{macrocode}
\DeclareOption{times}{\@timestrue}
\DeclareOption{arial}{\@arialtrue}
%    \end{macrocode}
% Transferência de todas as opções para a classe |report|
%    \begin{macrocode}
\DeclareOption*{\PassOptionToClass{\CurrentOption}{report}}
%    \end{macrocode}
% \clearpage
% Processamento de todas as opções
%    \begin{macrocode}
\ProcessOptions
%    \end{macrocode}
% Carregamento da classe |report|.
%    \begin{macrocode}
\LoadClass[12pt,a4paper]{report}
%    \end{macrocode}
% Execução de ações baseadas nas opções passadas
%    \begin{macrocode}
\if@times
  \RequirePackage{newtxtext}
  \RequirePackage{newtxmath}
  \RequirePackage[scaled=.90]{helvet}
  \RequirePackage{courier}
\fi
\if@arial
  \RequirePackage{newtxtext}
  \renewcommand{\familydefault}{\sfdefault}
\fi
%    \end{macrocode}
% Requisição dos pacotes necessários
%    \begin{macrocode}
\RequirePackage[brazil]{babel}
\RequirePackage[utf8]{inputenc}
\RequirePackage[T1]{fontenc}
\RequirePackage{indentfirst}
\RequirePackage{amsmath}
\RequirePackage{graphicx}
\RequirePackage{geometry}
\RequirePackage{fancyhdr}
\RequirePackage[hyphens]{url}
\RequirePackage[unicode,colorlinks]{hyperref}
%    \end{macrocode}

% Comandos para o arquivo de configuração
%    \begin{macrocode}
\newcommand*{\logo}              [1]{\gdef\@logo{#1}}
\newcommand*{\universidade}      [1]{\gdef\@universidade{#1}}
\newcommand*{\instituto}         [1]{\gdef\@instituto{#1}}
\newcommand*{\faculdade}         [1]{\gdef\@faculdade{#1}}
\newcommand*{\curso}             [1]{\gdef\@curso{#1}}
\newcommand*{\nivel}             [1]{\gdef\@nivel{#1}}
\newcommand*{\grau}              [1]{\gdef\@grau{#1}}
\newcommand*{\tipodetrabalho}    [1]{\gdef\@tipodetrabalho{#1}}
\newcommand*{\titulo}            [1]{\gdef\@titulo{#1}}
\newcommand*{\subtitulo}         [1]{\gdef\@subtitulo{#1}}
\newcommand*{\palavraschave}     [1]{\gdef\@palavraschave{#1}}
\newcommand*{\keywords}          [1]{\gdef\@keywords{#1}}
\newcommand*{\autor}             [2]{\gdef\@autor{#1}\gdef\@autorcitacao{#2}}
\newcommand*{\orientador}        [1]{\gdef\@orientador{#1}}
\newcommand*{\coorientador}      [1]{\gdef\@coorientador{#1}}
\newcommand*{\diretordafaculdade}[1]{\gdef\@diretordafaculdade{#1}}
\newcommand*{\cidade}            [1]{\gdef\@cidade{#1}}
\newcommand*{\estado}            [1]{\gdef\@estado{#1}}
\newcommand*{\ano}               [1]{\gdef\@ano{#1}}
\newcommand*{\membrodabanca}     [1]{\stepcounter{membros}\@membro{#1}}
%    \end{macrocode}

% Valores default para os comandos do arquivo de configuração
%    \begin{macrocode}
\universidade{Universidade Federal do Pará}
\tipodetrabalho{Trabalho de Conclusão de Curso}
\instituto{Instituto de Tecnologia}
\nivel{Graduação}
\cidade{Belém}
\estado{Pará}
\ano{\the\year}
\faculdade{NOME DA FACULDADE}
\curso{NOME DO CURSO}
\titulo{TÍTULO DO TRABALHO}
\palavraschave{PALAVRAS-CHAVE}
\keywords{KEYWORDS}
\autor{Nome Completo do Autor}{AUTOR, Nome Completo do}
\orientador{Titulação e Nome Completo do Orientador}
\grau{GRAU OBTIDO}
%    \end{macrocode}
% Comandos opcionais do arquivo de configuração
%    \begin{macrocode}
\logo{}
\subtitulo{}
\coorientador{}
\diretordafaculdade{}
%    \end{macrocode}

% ifs para os comandos opcionais
%    \begin{macrocode}
\newcommand\if@logo[2]{\if|\@logo\empty|#2\else#1\fi}
\newcommand\if@subtitulo[2]{\if|\@subtitulo\empty|#2\else#1\fi}
\newcommand\if@coorientador[2]{\if|\@coorientador\empty|#2\else#1\fi}
%    \end{macrocode}

% \begin{macro}{\capa}
%   Insere a capa formatada
%    \begin{macrocode}
\newcommand\capa{
  \thispagestyle{empty}
  \begin{center}
    \if@logo{\includegraphics[height=3cm]{\@logo}}{}
    \vspace{1cm}
    \par\MakeUppercase\@universidade
    \par\MakeUppercase\@instituto
    \par\MakeUppercase\@faculdade
    \vspace{4cm}
    \par\textbf{\MakeUppercase\@titulo}\if@subtitulo{:}{}
    \par\textbf{\@subtitulo}
    \vspace{4cm}
    \begin{flushright}
      \textbf{\MakeUppercase\@autor}
    \end{flushright}
    \vfill
    \MakeUppercase\@cidade{} -- \MakeUppercase\@estado
    \par\@ano
  \end{center}
  \clearpage
}
%    \end{macrocode}
% \end{macro}

% \begin{macro}{\folhaderosto}
%   Insere a folha de rosto formatada
%    \begin{macrocode}
\newcommand\folhaderosto{
  \thispagestyle{empty}
  \begin{center}
    \if@logo{\includegraphics[height=3cm]{\@logo}}{}
    \vspace{1cm}
    \par\MakeUppercase\@universidade
    \par\MakeUppercase\@instituto
    \par\MakeUppercase\@faculdade
    \vspace{2cm}
    \par\textbf{\MakeUppercase\@autor}
    \vspace{2cm}
    \par\textbf{\MakeUppercase\@titulo}\if@subtitulo{:}{}
    \par\textbf{\@subtitulo}
    \vspace{2cm}
    \begin{flushright}
      \begin{minipage}[right]{8cm}
        \@espacamento{1.5}{
          \fontsize{10pt}{10pt}\selectfont
          \@tipodetrabalho{} apresentado para obtenção
          do grau de \@grau{} em \@curso{}, do \@instituto{},
          da \@faculdade{}. Sob orientação de \@orientador{}%
          \if@coorientador{ e coorientação de \@coorientador{}}{}.
        }
       \end{minipage}
    \end{flushright}
    \vfill
    \MakeUppercase\@cidade{} -- \MakeUppercase\@estado
    \par\@ano
  \end{center}
  \clearpage
}
%    \end{macrocode}
% \end{macro}

% \begin{macro}{\fichacatalografica}
%   Insere a ficha catalográfica formatada
%    \begin{macrocode}
\newcommand\fichacatalografica[3]{
  \thispagestyle{empty}
  \newcommand\@biblioteca{#1}
  \newcommand\@classificacao{#2}
  \newcommand\@cdd{#3}
  \null\vfill
  \begin{center}\sf
    Dados Internacionais de Catalogação na Publicação (CIP)\par
    \@biblioteca / \@universidade{}, \@cidade{}-\@estado{}\par
    \rule{1\linewidth}{0.5pt}\par\vspace*{0.3cm}
    \begin{minipage}{0.95\linewidth}
      {
        \footnotesize\setlength{\baselineskip}{1.1\baselineskip}
        \@autorcitacao{}.
        \par\setlength{\parindent}{0.75cm}
        \@titulo{}\if@subtitulo{: \MakeLowercase\@subtitulo{}}{} /
        \@autor{}; orientador \@orientador{} --- \@ano{}.
        \par\vspace*{0.4cm}
        \@tipodetrabalho{} (\@nivel{} em \@curso{}) -- \@universidade{},
        \@instituto{}, Curso de \@curso{}, \@cidade{}, \@ano{}.
        \par\vspace*{0.4cm}
        \@classificacao{}
        \par\vspace*{0.4cm}
        \begin{flushright}
          CDD - \@cdd{}
        \end{flushright}
      }
    \end{minipage}
    \par\vspace*{0.3cm}\rule{1\linewidth}{0.5pt}
  \end{center}
  \clearpage
}
%    \end{macrocode}
% \end{macro}

% Lista de membros da banca avaliadora
%    \begin{macrocode}
\newcounter{membros}
\long\gdef\@membro#1{\membro@add#1,,\end}
\long\gdef\membro@add#1,#2,#3,#4\end{
  \expandafter\def\csname membro@\alph{membros}@nome\endcsname{\expandafter#1}
  \expandafter\def\csname membro@\alph{membros}@instituicao\endcsname{#2}
  \ifx,#3,\expandafter\let\csname membro@\alph{membros}@papel\endcsname\empty
  \else \expandafter\def\csname membro@\alph{membros}@papel\endcsname{#3}\fi
}
%    \end{macrocode}

% \begin{macro}{\folhadeaprovacao}
%   Insere a folha de aprovação formatada
%    \begin{macrocode}
\newcommand\folhadeaprovacao[1]{
  \thispagestyle{empty}
  \begin{center}
    \textbf{\MakeUppercase\@titulo}\if@subtitulo{:}{}
    \par\textbf{\@subtitulo}
  \end{center}
  \@indentacao{0}{\@espacamento{2.0}{
    Este trabalho foi julgado adequado em #1
    para a obtenção do Grau de \@grau{} em \@curso{},
    aprovado em sua forma final pela banca examinadora
    que atribuiu o conceito $\underline{ \qquad \qquad \qquad}$.
  }}
  \par\vfil
  \begin{flushright}
    \begin{minipage}[b]{10cm}
      \begin{flushright}
        \@espacamento{1.25}{
          \newcounter{tmp}
          \c@tmp=0
          \loop
            \advance\c@tmp by 1
            \vspace*{1.5cm}\par
            \if
              \expandafter\csname membro@\alph{tmp}@nome\endcsname\empty
            \else
              \vrule width 10cm height 0.2mm\par
            \fi
            {\large\expandafter\csname membro@\alph{tmp}@nome\endcsname{}}\par
            \if
              \expandafter\csname membro@\alph{tmp}@papel\endcsname\empty
            \else
              {\large{\expandafter\csname membro@\alph{tmp}@papel\endcsname}}\par
            \fi
            {\textsc{\expandafter\csname
            membro@\alph{tmp}@instituicao\endcsname}}\par
            \ifnum\c@tmp<\c@membros
          \repeat
          \if \@diretordafaculdade\empty\else
            \vspace{2cm}\par
            \vrule width 10cm height 0.2mm\par
            {\large\@diretordafaculdade{}}\par
            {\small\MakeUppercase{Diretor da \@faculdade}}
          \fi
        }
      \end{flushright}
    \end{minipage}
  \end{flushright}
  \clearpage
}
%    \end{macrocode}
% \end{macro}

% \begin{macro}{\referencias}
%   Insere a lista de referências
%    \begin{macrocode}
\newcommand\referencias[2][Referências Bibliográficas]{
  \renewcommand{\bibname}{#1}
  {
    \setlength{\baselineskip}{1.2\baselineskip}
    \bibliography{#2}
    \addcontentsline{toc}{chapter}{\bibname}
  }
}
%    \end{macrocode}
% \end{macro}

% \begin{macro}{\@referencia}
%   Comando interno que armazena a referência bibliográfica do trabalho
%    \begin{macrocode}
\newcommand\@referencia{
  \@autorcitacao{}. \textbf{\@titulo}\ignorespaces
  \if@subtitulo{: \MakeLowercase\@subtitulo{}}{}.
  \@tipodetrabalho{} (\@nivel{} em \@curso{}) ---
  \@universidade{}, \@cidade{}, \@ano{}.\par\vspace{1cm}
}
%    \end{macrocode}
% \end{macro}

% \clearpage
% \begin{macro}{\@indentacao}
%   Comando interno para alterar o recuo de primeira linha
%    \begin{macrocode}
\newcommand\@indentacao[2]{{\setlength{\parindent}{#1cm}#2}}
%    \end{macrocode}
% \end{macro}

% \begin{macro}{\@espacamento}
%   Comando interno para alterar o espaçamento entrelinhas
%    \begin{macrocode}
\newcommand\@espacamento[2]{{
  \def\baselinestretch{#1}
  \@currsize #2\par
}}
%    \end{macrocode}
% \end{macro}

% \begin{environment}{dedicatoria}
%   Insere a dedicatória com fonte em itálico
%    \begin{macrocode}
\newenvironment{dedicatoria}
{
  \thispagestyle{empty}
  \null\vfill
  \begin{flushright}
    \begin{minipage}{10cm}
      \begin{flushright}
        \fontshape{it}\selectfont
}{
      \end{flushright}
    \end{minipage}
  \end{flushright}
  \clearpage
}
%    \end{macrocode}
% \end{environment}

% \begin{environment}{agradecimentos}
%   Insere os agradecimentos
%    \begin{macrocode}
\newenvironment{agradecimentos}
{
  \thispagestyle{empty}
  \begin{center}
    \huge\bfseries Agradecimentos
  \end{center}
  \par\vspace{1cm}\setlength{\parindent}{1cm}
}{\clearpage}
%    \end{macrocode}
% \end{environment}

% \begin{environment}{epigrafe}
%   Insere a epígrafe
%    \begin{macrocode}
\newenvironment{epigrafe}
{
  \thispagestyle{empty}
  \null\vfill
  \begin{flushright}
    \begin{minipage}{10cm}
      \begin{flushright}
        \fontshape{it}\selectfont
}{
      \end{flushright}
    \end{minipage}
  \end{flushright}
  \clearpage
}
%    \end{macrocode}
% \end{environment}

% \begin{environment}{resumo}
%   Insere o resumo com as palavras-chave
%    \begin{macrocode}
\newenvironment{resumo}
{
  \thispagestyle{empty}
  \begin{center}
    \huge\bfseries Resumo
  \end{center}
  \par\vspace{1cm}\setlength{\parindent}{0cm}
  \@referencia{}
}{
  \par\vspace{0.25cm}
  {\bfseries Palavras-chave:} \@palavraschave{}.
  \clearpage
}
%    \end{macrocode}
% \end{environment}

% \begin{environment}{abstract}
%   Insere o abstract com as keywords
%    \begin{macrocode}
\renewenvironment{abstract}
{
  \thispagestyle{empty}
  \begin{center}
    \huge\bfseries Abstract
  \end{center}
  \par\vspace{1cm}\setlength{\parindent}{0cm}
  \@referencia{}
}{
  \par\vspace{0.25cm}
  {\bfseries Keywords:} \@keywords{}.
  \clearpage
}
%    \end{macrocode}
% \end{environment}

% Formato das páginas
%    \begin{macrocode}
\geometry{a4paper,
          top=3cm,bottom=2cm,left=3cm,right=2cm,
          headsep=5mm,headheight=7mm,
          marginparsep=5mm,marginparwidth=10mm}
%    \end{macrocode}

% Formato do cabeçalho
%    \begin{macrocode}
\renewcommand{\headrulewidth}{0pt}
\renewcommand{\footrulewidth}{0pt}
\fancypagestyle{plain}{\fancyhf{}} % for chapters first pages
\fancypagestyle{numbered}{
  \fancyhf{}
  \fancyhead[R]{\thepage}
}
%    \end{macrocode}

% Configurações iniciais para o trabalho
%    \begin{macrocode}
\AtBeginDocument{
  \flushbottom
  \pagestyle{numbered}
  \setlength{\parindent}{1.75cm}
  \setlength{\parskip}{1ex plus 0.5ex minus 0.2ex}
  \input{\jobname.cfg}
  \hypersetup{
    hypertexnames,bookmarksopen=true,
    linkcolor=black,citecolor=black,urlcolor=black,
    pdftitle={\@titulo\if@subtitulo{: \@subtitulo}{}},
    pdfauthor={\@autor},
    pdfsubject={\@tipodetrabalho},
    pdfkeywords={\@palavraschave},
    pdfcreator={LaTeX class ufpatcc with hyperref package}
  }
}

\endinput
%</class>
%    \end{macrocode}
%\Finale
